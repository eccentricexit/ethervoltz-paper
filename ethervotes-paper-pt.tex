\documentclass[
	% -- opções da classe memoir --
	article,			% indica que é um artigo acadêmico
	11pt,				% tamanho da fonte
	oneside,			% para impressão apenas no verso. Oposto a twoside
	a4paper,			% tamanho do papel. 
	% -- opções da classe abntex2 --
	%chapter=TITLE,		% títulos de capítulos convertidos em letras maiúsculas
	%section=TITLE,		% títulos de seções convertidos em letras maiúsculas
	%subsection=TITLE,	% títulos de subseções convertidos em letras maiúsculas
	%subsubsection=TITLE % títulos de subsubseções convertidos em letras maiúsculas
	% -- opções do pacote babel --
	english,			% idioma adicional para hifenização
	brazil,				% o último idioma é o principal do documento
	sumario=tradicional
	]{abntex2}


% ---
% PACOTES
% ---

% ---
% Pacotes fundamentais 
% ---
\usepackage{lmodern}			% Usa a fonte Latin Modern
\usepackage[T1]{fontenc}		% Selecao de codigos de fonte.
\usepackage[utf8]{inputenc}		% Codificacao do documento (conversão automática dos acentos)
\usepackage{indentfirst}		% Indenta o primeiro parágrafo de cada seção.
\usepackage{nomencl} 			% Lista de simbolos
\usepackage{color}				% Controle das cores
\usepackage{graphicx}			% Inclusão de gráficos
\usepackage{microtype} 			% para melhorias de justificação
% ---
		
% ---
% Pacotes adicionais, usados apenas no âmbito do Modelo Canônico do abnteX2
% ---
\usepackage{lipsum}				% para geração de dummy text
% ---
		
% ---
% Pacotes de citações
% ---
\usepackage[brazilian,hyperpageref]{backref}	 % Paginas com as citações na bibl
\usepackage[alf]{abntex2cite}	% Citações padrão ABNT
% ---

% ---
% Configurações do pacote backref
% Usado sem a opção hyperpageref de backref
\renewcommand{\backrefpagesname}{Citado na(s) página(s):~}
% Texto padrão antes do número das páginas
\renewcommand{\backref}{}
% Define os textos da citação
\renewcommand*{\backrefalt}[4]{
	\ifcase #1 %
		Nenhuma citação no texto.%
	\or
		Citado na página #2.%
	\else
		Citado #1 vezes nas páginas #2.%
	\fi}%
% ---

% ---
% Informações de dados para CAPA e FOLHA DE ROSTO
% ---
\titulo{EtherVotes: Blockchain e Canais de Estado\\Para Eleições Auditáveis}
\autor{Matheus Faria de Alencar}
\local{Brasil}
\data{\today}
% ---

% ---
% Configurações de aparência do PDF final

% alterando o aspecto da cor azul
\definecolor{blue}{RGB}{41,5,195}

% informações do PDF
\makeatletter
\hypersetup{
     	%pagebackref=true,
		pdftitle={\@title}, 
		pdfauthor={\@author},
    	pdfsubject={EtherVotes: Blockchain e Canais de Estado Para Eleições Auditáveis},
	    pdfcreator={LaTeX with abnTeX2},
		pdfkeywords={ethervotes}{blockchain}{ethereum}{raiden}{eleições}, 
		colorlinks=true,       		% false: boxed links; true: colored links
    	linkcolor=blue,          	% color of internal links
    	citecolor=blue,        		% color of links to bibliography
    	filecolor=magenta,      		% color of file links
		urlcolor=blue,
		bookmarksdepth=4
}
\makeatother
% --- 

% ---
% compila o indice
% ---
\makeindex
% ---

% ---
% Altera as margens padrões
% ---
\setlrmarginsandblock{3cm}{3cm}{*}
\setulmarginsandblock{3cm}{3cm}{*}
\checkandfixthelayout
% ---

% --- 
% Espaçamentos entre linhas e parágrafos 
% --- 

% O tamanho do parágrafo é dado por:
\setlength{\parindent}{1.3cm}

% Controle do espaçamento entre um parágrafo e outro:
\setlength{\parskip}{0.2cm}  % tente também \onelineskip

% Espaçamento simples
\SingleSpacing

% ----
% Início do documento
% ----
\begin{document}

% Retira espaço extra obsoleto entre as frases.
\frenchspacing 

% ----------------------------------------------------------
% ELEMENTOS PRÉ-TEXTUAIS
% ----------------------------------------------------------


%---
% página de titulo
\maketitle

% resumo em português
\begin{resumoumacoluna}
Os problemas segurança de processos eleitorais serviram como combustível para uma série de estudos e avanços tecnologicos para melhorar detectabilidade de fraudes e ao mesmo tempo prover celeridade na apuração de votos. Estes avanços acarretaram em um aumento de complexidade dos processos e sistemas, que por consequencia colocaram barreiras de tempo e custo em processos de auditoria além de centralizarem o poder de condução das mesmas no administrador. {P1: Definir custos, centralização de poder, dificuldades de auditoria}

A utilização de sistemas cliente-servidor também não são soluções viáveis para a condução de processos eleitorais já que oferecem pontos de falha vulneraveis a ataques de negação de serviço e dependem da confiança nos envolvidos no processo para adminstração das chaves. {P2: Apresentar problema de soluções com servidores}

Este artigo documenta uma prova de conceito de sistema eleitoral chamado EtherVoltz (pronunciado Íter vÔltz) que não depende de software e faz uso do blockchain de um computador global decentralizado (1) e distribuído (2) para baratear e simplificar os processos de eleição e auditoria. Esta proposta também remove do administrador a tarefa e o poder de controlar os registros digitais dos votos após a votação sobrando ao mesmo a tarefa do gerenciamento dos votos impressos para posteriores auditorias. {P3: Apresentar EtherVotes como possível solução para o problema} 

\end{resumoumacoluna}

\renewcommand{\resumoname}{Abstract}
\begin{resumoumacoluna}
	\begin{otherlanguage*}{english}
		Security issues in election processes fueled a series of worldwide studies and technological advances to improve fraud detectability and speed up the vote counting process. These advances caused an increase in the complexity of the electoral process and as consequence placed time and cost barriers on the auditing process as well as a concentration of powers in the administrator.
		
		The use of the client-server architecure has also been proven to be challenging since they are vulnerable to denial-of-service attacks and still rely on trusting the ones involved on the processes to manage the keys that guarantee its safety.
		
		This article documents a proof of concept of an electoral system called EtherVoltz that does is software-independent and leverages the blockchain of global computer that is decentralized and distributed to simplify and cheapen the election and auditing processes. This proposal also removes from the administrator, the burden and power of managing the digital records that are generated during the election process, leaving the task of managing the paper trails generated.
		
		\vspace{\onelineskip}
		
		\noindent
     	\textbf{Palavras-chaves}: ethervotes. blockchain. ethereum. eleições. raiden.
	\end{otherlanguage*}  
\end{resumoumacoluna}

% ----------------------------------------------------------
% ELEMENTOS TEXTUAIS
% ----------------------------------------------------------
\textual

 ----------------------------------------------------------
% Seção de explicações
% ----------------------------------------------------------
\section{Introdução}

A medida componentes eletrônicos foram sendo barateados e miniaturizados, sistemas eletrônicos para votação vem sendo desenvolvidos para aumentar a velocidade de apuração de votos e através de soluções com registros de voto digital. A primeira geração destas urnas são as chamadas dependentes de software e utilizam apenas o registro digital de votos ou DRE e é o sistema que o Brasil utiliza desde a sua concepção 1996. {P1: Introdução à area}

De fato, a urna brasileira de primeira geração em uso até 2018, utiliza apenas o registro digital de voto e não gera nenhum documento auditável pelo eleitor e portanto não adere ao princípio da independência de software (3). Não possibilita aos representantes da sociedade conferir e auditar o resultado da apuração eletrônica dos votos. Foi rejeitada por todos os mais de 50 países que a avaliaram [1] que optaram por soluções que utilizam documento auditável como urnas de segunda e terceira geração. {P2: Motivação} 

\subsection{Custos}

A Reforma Eleitoral de 2015 reintroduz o voto impresso ao processo eleitoral brasileiro que propiciona independência de software ao processo. O preço de cada urna em 2018 segundo estimativas do TSE sobe de 600 doláres para 800 dolares com a utilização de impressoras, sendo necessárias aquisições de mais de 830 mil impressoras e mais de 400 mil novas urnas segundo o TSE [2].  Além das dispesas relacionadas a aquisição das urnas, o registro digital de voto das urnas requer um amplo esquema de segurança envolvendo as forças armadas, para realizar o transporte dos registros de voto digitais e das urnas. {P3: Problemas}

Uma solução trivial, porém falha para baratear o controle dos registros digitais muitas vezes proposto é a construção de um sistema cliente-servidor para a coleta e apuração dos dados. Três problemas cruciais desta solução são: 
1. Torna servidores alvos extremamente valiosos a ataques externos e malwares. Os custos para a proteção destes sistemas sobe proporcionalmente a complexidade deles.
2. Elevado custo de manutenção dos servidores por necessidade de redundância para proteção contra ataques ataques de negação de serviço (4).
3. Deposita uma enorme quantidade de confiança em todos envolvidos processo de produção e manutenção do software, o que torna o sistema sucetível a ataques internos indetectáveis. 
No EtherVotes estes problemas são resolvidos ao delegar a tarefa de garantir a integridade, confiabilidade e disponibilidade dos dados a um blockchain em um computador global formado por uma rede p2p chamado Ethereum. Mais detalhes serão descritos na seção 5 deste documento.  {P3: Apresentar dificuldades financeiras do uso bases de dados comuns devido a vulnerabilidade a ataques DDoS, ataques internos.}

\subsection{Centralização}

 P1: Apresentar resistência e condições de auditoria das eleições de 2014
 
 P2: Apresentar ethervotes como proposta para barateamento das eleições através de uma aplicação distribuida, a transformação do voto em uma criptomoeda infracionável e um minicomputador (como o raspberrypi) como interface com a urna.
 
 P3: Dificuldades de auditorias de fraude devido a centralização do poder
 
 P4: Experimentos desenvolvidos
 
 P5: Organização do texto
 
 Nota 3: Um sistema eleitoral é independente do software se uma modificação ou erro não-detectado no seu software não pode causar uma modificação ou erro indetectável no resultado da apuração. [3]

\section{Motivação: Eleições Caras e Opacas}
P1: Apresentar altos custos das eleições em 
\subsection{Eleições de Primera Geração e Seus Custos}
P1: Apresentar quantos países utilizam a urna DRE sem vice e a opinião internacional sobre ela.

P2: Apresentar o caso Marília, o caso Itajaí, o caso Diadema e o caso Alagoas

P3: Apresentar custo de urnas DRE sem e com VICE

P4: Concluir introduzindo a necessidade de VICE e conceito de independência de software, e proposta do trabalho para baratear e aumentar a segurança de sistemas  eleitorais com VICE ao mover a urna à máquina virtual ethereum.

\section{Informações Contextuais}
 P1: Apresentar a maquina virtual ethereum como um computador global e a linguagem de programação solidity
 
 P2: Apresentar desafios de escalabilidade atuais e o conceito de canais de estado
 
\section{Estado da Arte}
 P1: Apresentar desafios de escalabilidade do blockchain citando e-vox e follow-my-vote
 
 P2: Apresentar problemas de auditabilidade de soluções dependentes de software
 
\section{EtherVotes}
P1: Descrever que os objetivos da proposta são delegar a responsabilidade da integridade e disponibilidade dos votos à EVM

\subsection{Arquitetura e Metodologia}
P1: Apresentar problemas de soluções baseadas apenas em blockchain (tempo de transação, limitações da evm)

P2: Apresentar canais de estado (Rede raiden) como solução

P3: Concluir com a criação de VoteToken.sol e o procedimento de transferência de tokens para carteiras de candidatos.

P4: Notar que estes contratos são simples, ficam disponíveis no enderenço na EVM após deployment e seus códigos-fonte podem e devem ser publicados para inspeção independente.

\subsection{VoteToken}
P1: Descrever o procedimento da emissão de moedas fazendo um paralelo ao bitcoin.

P2: Notar sobre a vantagem de segurança contra ataques internos, já que só possuem poder de voto carteiras que receberam votetokens que o STE emitiu. 

\subsection{Controle de carteiras}
P1: Descrever o contrato MachineRegistry.sol como solução para exposição e controle de carteiras (chaves públicas)

P2: Descrever efeito da impressão da carteira da urna nas boletas contra fraudes

\subsection{Apuração de Votos}
P1: Descrever o que ocorre após as eleições no modelo antigo. Forças armadas, logística e transporte das urnas e boletas para apuração dos votos, perigo de fraude através da destruição de evidências (destruir urnas e boletas não favoráveis.

P2: Descrever que no ethervotes os votos estão registrados no blockchain, por isso não há necessidade de segurança no transporte do equipamento eletrônico. Na verdade, o mesmo poderiam ou deveriam ser destruídas para evitar o vazamento de chaves privadas.

P3: Descrever o processo de apuração atual e autoridade responsável.

P4: Descrever o processo de apuração no ethervotes (getBalance)
\subsection{Auditorias}
P1: Descrever processo de auditoria de uma única urna no ethervotes. 

1-Solicitação de todas as boletas com a chave pública utilizada na urna a ser auditada (Ainda envolve o TRE)

P1: Explicar que não há necessidade de solicitar acesso a urnas, ou código fonte.

2-Solicitar um histórico de todas as transações realizadas por aquela chave pública à maquina virtual ethereum (não envolve o TSE)

P1: Explicar que o estado não tem controle sobre os dados digitalizados das urnas.

3-Comparar boletas entregues pelo TRE com as transações registradas no blockchain. 

\section{Conclusão}
  P1: Celeridade e Transparência nas auditorias
  
  P2: Decentralização do poder
  
  P3: Barateamento das eleições
  
  
\section{Experimentos}

  P1: Apresentar prova de conceito construida e resultados
  
  
% ----------------------------------------------------------
% Referências bibliográficas
% ----------------------------------------------------------
\bibliography{abntex2-modelo-references}


\end{document}
